%%%%%%%%%%%%%%%%%%%%%%%%%%%%%%%%%%%%%%%%%%%%%%%%%%%%%%%%%%%%%%%%%%%%%%%%%%%%%%%%
%2345678901234567890123456789012345678901234567890123456789012345678901234567890
%        1         2         3         4         5         6         7         8

\documentclass[letterpaper, 10 pt, conference]{ieeeconf}  % Comment this line out if you need a4paper
%\documentclass[a4paper, 10pt, conference]{ieeeconf}      % Use this line for a4 paper
\IEEEoverridecommandlockouts                              % This command is only needed if 
                                                          % you want to use the \thanks command
\overrideIEEEmargins                                      % Needed to meet printer requirements.

% See the \addtolength command later in the file to balance the column lengths
% on the last page of the document

% The following packages can be found on http:\\www.ctan.org
\usepackage{graphics} % for pdf, bitmapped graphics files
\usepackage{epsfig} % for postscript graphics files
\usepackage{mathptmx} % assumes new font selection scheme installed
\usepackage{times} % assumes new font selection scheme installed
\usepackage{amsmath} % assumes amsmath package installed
\usepackage{amssymb}  % assumes amsmath package installed
\usepackage{url}

\title{\LARGE \bf
Human Learning vs. Deep Learning for Atari Games
}

\author{Daniel Seita$^{1}$% <-this % stops a space
\thanks{$^{1}$University of California, Berkeley, USA. {\tt\small seita@berkeley.edu}}%
}

\begin{document}

\maketitle
\thispagestyle{empty}
\pagestyle{empty}

%%%%%%%%%%%%%%%%%%%%%%%%%%%%%%%%%%%%%%%%%%%%%%%%%%%%%%%%%%%%%%%%%%%%%%%%%%%%%%%%
\begin{abstract}

The combination of reinforcement learning and deep learning has led to impressive results in
training AI agents to play Atari games. Nonetheless, there is still much room for improvement,
especially for Atari games that require substantial long-term strategy, and the entire process of
how and what neural networks learn is poorly understood. In this paper, we augment the current
understanding of neural network based agents by collecting data from humans playing Atari games. We
use knowledge derived from human game logs to present a different initial exploration policy for AI
agents to get faster and more accurate state-action value approximations. In addition, we also
collect various quantitative and qualitative data from the human game play. In doing so, we aim to
make contributions to the cognitive science process of how humans learn. Our results imply that
using human data for exploration policies can be beneficial, and that humans learn a variety of
``optimal'' game strategies despite having trouble with coordination of the game controls.

\end{abstract}


%%%%%%%%%%%%%%%%%%%%%%%%%%%%%%%%%%%%%%%%%%%%%%%%%%%%%%%%%%%%%%%%%%%%%%%%%%%%%%%%
\section{Introduction}\label{sec:intro}

Deep learning has become one of the most popular and fastest growing research subfields of
artificial intelligence, as evident by the superior performance of (deep) neural networks over
classical methods in realms such as speech recognition~\cite{speech} and image
classification~\cite{image_classification}.

It has also been shown that deep learning can be used for challenging tasks in reinforcement
learning, where the job of an AI agent is not to perform ``simple'' classification, but to learn
from high-dimensional, correlated data with a scalar reward signal that is noisy and may exhibit
complicated, long-term rewards. For instance,~\cite{mnih-atari-2013} combined model-free
reinforcement learning with deep learning techniques to develop an AI agent capable of learning how
to play several Atari 2600 games at a level matching or exceeding human performance. The AI only
learned from the game frames and the score, just like how a human would learn.

Nonetheless, despite the progress advanced by neural networks, many questions still remain about how
exactly neural networks learn, and it is still unclear if this underlying ``process'' is at all
similar to the way that humans would learn~\cite{szegedy2013intriguing,nguyen2015deep}.  One way to
bridge the knowledge gap between the learning process of neural network agents and humans is to
investigate how humans would learn to play these Atari games. In this report, we investigate the
numerous research questions derived from human learning.

First, it might be possible to augment the learning process of neural network agents with human data
to accelerate training to get fast, high-quality policies. We train a classifier from game frames to
actions based on human data, and show that one can incorporate the classifier during the exploration
phase of the neural network agent, when it is following a $(1-\epsilon)$-greedy policy with high
$\epsilon$ (i.e., close to one). Rather than have the ``$\epsilon$ cases'' correspond to
\emph{random} actions, the AI agent can use those cases to follow the \emph{human action}.

Second, it is worthy to try and directly understand the human learning process. To do this, we
design a task on Amazon Mechanical Turk, where players must play three Atari games (Pong, Breakout,
and Space Invaders) and answer well-designed survey questions. Furthermore, we use different game
orderings to test the possibility that transfer learning improves game performance.  We report on
the quantitative and qualitative results of the survey and make connections with the neural network
based agents, hoping to ultimately better understand the human learning and deep learning processes
that enable the corresponding agents to successfully play Atari games.


%%%%%%%%%%%%%%%%%%%%%%%%%%%%%%%%%%%%%%%%%%%%%%%%%%%%%%%%%%%%%%%%%%%%%%%%%%%%%%%%
\section{Background and Related Work}\label{sec:related_work}

This report builds upon the groundbreaking work of~\cite{mnih-atari-2013} and the subsequent
follow-up~\cite{mnih-dqn-2015}. Both describe a neural network agent that can successfully
learn control policies for Atari 2600 games. The Arcade Learning
Environment~\cite{bellemare13arcade} provides the necessary software to test and benchmark agents on
Atari games, which are a relatively complicated reinforcement learning environment and sufficiently
diverse enough such that an agent that can play well in a variety of games is likely to have learned
without hand-engineered features.

To train the agent, they use a variant of Q-learning~\cite{Sutton_1998}. In standard Q-Learning for
solving a Markov Decision Process, one has state-action values $Q(s,a)$ for state $s$ and action
$a$. This is the expected sum of discounted rewards for the agent starting at state $s$, taking
action $a$, and from then on, playing optimally according to the action determined by the policy.
With Atari games, the states are \emph{sequences} of game frames $x_1,x_2,\ldots,x_t$ encountered
during game play\footnote{Technically,~\cite{mnih-atari-2013} reports that states are sequences of
game frames \emph{and} actions: $x_1,a_1,x_2,\ldots,a_t$. When doing Q-Learning, however, their code
only considers four consecutive frames and, to the best of our knowledge, does not take into account
actions other than the current one under consideration.}. The optimal action-value function $Q$
obeys the \emph{Bellman equation} identity: 

\begin{equation}\label{eq:bellman}
Q(s,a) = \mathbb{E}_{s'}\left[r + \gamma \cdot \max_{a'} Q(s',a') \mid s,a \right].
\end{equation}

The process of Q-Learning (or more generally, reinforcement learning) is to estimate the Q-values
using the Bellman equation as an iterative update.

The states are extremely high dimensional; even with downsampling, one frame is an $(84\times
84)$-dimensional input, and storing all $Q(s,a)$ values explicitly in a table is impractical.
Therefore, the $Q(s,a)$ values are \emph{approximated} by a neural network parameterized by its
weights $\theta$, and it is $\theta$ that the Q-Learning algorithm must learn.

In practice,~\cite{mnih-dqn-2015} use a variant of online Q-Learning (with a $(1-\epsilon)$-greedy
policy for exploration) tailored to the task of Atari games.  The most important differences are:

\begin{itemize}
    \item They use \emph{experience replay} to store the agent's samples $(s_t,a_t,r_t,s_{t+1})$ at
    each time step in a dataset. When the Q-Learning algorithm updates $\theta$, it applies updates
    to a subset of the data. This is in contrast with standard Q-learning which only performs one
    update for each sample. Advantages include greater data efficiency and breaking the correlation
    among consecutive samples.
    \item They use a separate network for generating the targets (i.e., the $r + \gamma \cdot
    \max_{a'}Q(s',a')$ term in Equation~\ref{eq:bellman}), for the purposes of increasing the
    algorithm's numerical stability.
\end{itemize}

The DQN trained with this variant of Q-Learning was able to excel at many Atari games, especially
fast-paced games with simple rules such as Breakout. It was, however, weak on games such as
Montezuma's Revenge, which requires substantial long-term strategy.

Recently, there has been a surge of follow-up work for training agents to play Atari games.
In~\cite{nips-atari-2014}, they augment training using data collected \emph{offline} through the use
of Monte-Carlo tree search planning. The ``offline player,'' while substantially better than DQN,
cannot play in real time, but can be used to improve DQN's performance. The work
of~\cite{schaul2015prioritized} introduces prioritized experience replay to train DQN agents
faster since the most important transitions would be considered more frequently. This is a variant
of \emph{prioritized sweeping}~\cite{Moore93prioritizedsweeping}, a well-known technique used in
planning algorithms to pick the states to update in order of priority.
In~\cite{stadie2015incentivizing}, they propose more sophisticated exploration policies to improve
performance on Atari games.

In~\cite{wang2015dueling}, they present a different neural network architecture specialized for
reinforcement learning, and~\cite{van2015deep} proposes the Double DQN, which mitigates the problem
of the ``max'' operator using the same values to both select and evaluate an action (thus leading to
overoptimistic value estimates). At the time of publication, it was the highest-quality DQN
available. Finally, some follow-up work does not take a direct focus towards improving the baseline
DQN. For instance,~\cite{tampuu2015multiagent} extends the DQN architecture to \emph{multiagent
environments} and focuses on what happens when two independent DQN agents are trained for the game
of Pong.

While there has been much work concerning the technical aspects of DQN and its variants, there has
not been comparable interest in using data from human game logs. There is, however, a large
literature in the cognitive science field about human reasoning and learning~\cite{stenninghuman}.
The aim of this work is to bridge the gap between cognitive science and DQN by analyzing human game
logs and understanding the connection between human learning versus deep learning.


%%%%%%%%%%%%%%%%%%%%%%%%%%%%%%%%%%%%%%%%%%%%%%%%%%%%%%%%%%%%%%%%%%%%%%%%%%%%%%%%
\section{Design of Experiments}\label{sec:experiments}

We present experiments that achieve the dual goals of (1) describing how humans learn to play Atari
games, and (2) using the human game log to augment the training process of deep Q-networks.

\subsection{Amazon Mechanical Turk}\label{ssec:human_experiment}

\begin{figure}[t]
\centering
%\fbox{\rule[-.5cm]{0cm}{4cm} \rule[-.5cm]{4cm}{0cm}}
\includegraphics[width=3in]{fig_atari_games}
\caption{The three Atari games we used in this experiment: Pong (left), Breakout (middle), and Space
Invaders (right). Pong and Breakout are similar in that both use a paddle to hit a ball that bounces
in the game space. In Space Invaders, the player controls the ship at the bottom and aims to shoot
at all the other ships.}
\label{fig:atari}
\end{figure}

To gather human data in a principled, programmatic way, we implemented a task for workers on Amazon
Mechanical Turk (AMTurk), which is an online service where people can do tasks in exchange for
payment. For our AMTurk task, we had players play three of the Atari games: Pong, Breakout, and
Space Invaders. Figure~\ref{fig:atari} shows example screenshots of the three games.

We selected these three for several reasons. First, all three games are commonly used as benchmarks
for DQN agents; the original paper that introduced DQN used these three
games~\cite{mnih-atari-2013}. Second, Pong and Breakout are similar because players control a
paddle and can move it in one axis (vertically for Pong, horizontally for Breakout) and have to
hit a moving, bouncing ball. Both games require skillful coordination to hit the
ball at the correct locations and momentum to get the balls to go in desired directions.
Consequently, Pong and Breakout may test the effect of transfer learning, which we can evaluate by
checking the game performances of those players on Pong with or without Breakout experience (so long
as we have sufficient players to reduce the variance with few players). Finally, Space Invaders is a
game that DQN performs reasonably well on, but still often loses to human players, and requires
long-term strategy. It might be worthwhile to see if human data can improve DQN's performance on
Space Invaders.

For the AMTurk experiment, each player was assigned one of the six possible game orderings so that
we can evaluate transfer learning. The players played each game for exactly ten minutes. To motivate
the players to play and get high scores (and to avoid just letting the timer run out without doing
anything) we set three payment thresholds based off of performance. For each game, players earned
\$1.00, \$1.50, and \$2.00 for clearing the first, second, and third thresholds, respectively. All
thresholds were based off of game scores. For instance, Breakout's thresholds were set at 5, 30, and
90 points for one game episode. The first two thresholds are easy enough for almost all new players
to play, while the third would require some additional skill or be attainable only for fast
learners or experienced players.

We told the players which controls to use (the four arrow keys, plus the space bar), but did not
tell them what they did, with the exception of telling them that the space bar in Breakout allowed
the ball to appear. We did this because a pilot run indicated that confusion among the space bar was
a top issue.

We also make the players answer various survey questions to collect qualitative data. We had a set
of post-game questions that players had to answer right after they played a game. The template was:

\begin{itemize}
    \item A Likert question: \emph{I played well.}
    \item A Likert question about a game-specific strategy:
    \begin{itemize}
        \item Pong: \emph{It is a good strategy to try and make the ball take an angled path towards the opponent.}
        \item Breakout: \emph{It is a good strategy to try and clear the board by eliminating one row at a time.}
        \item S.I.: \emph{It is a good strategy to try and eliminate ships in one column before moving on to another column.}
    \end{itemize}
    \item (Free resp.) \emph{What was the final strategy you decided on?}
    \item (Free resp.) \emph{Was there anything from a previous game in this HIT\footnote{A HIT in AMTurk stands
    for ``Human Intelligence Task.''} that helped?}
\end{itemize}

The purpose of the game-specific Likert scale question is to get the players to think about good
general strategies for this game.

This template of post-game survey questions was asked directly after a game, so players did the
following in succession: Game 1, post-game survey, Game 2, post-game survey, Game 3, post-game
survey, and then finalized their task with some general survey questions at the end. At the very
end, we asked the players if they think they now knew effective strategies for all three games and
how they would rank their level of expertise in each game (both of these were on a Likert scale).
Our final question explicitly asked them if they thought they had developed good strategies to play
all three games. We discuss our results in Section~\ref{ssec:human_results}.

\textbf{TODO Well at least I have something here ... revise it and polish it.}

\subsection{Deep Q-Networks}\label{ssec:dqn_experiment}

While the human survey data and game scores can be used to gain various insights, it is also
interesting to investigate whether it is useful to incorporate human data in the process of training
a deep Q-network (DQN). In this section, we discus two applications of using the human data. The
first would be to change the exploration policy of DQN so that, instead of being a standard
($1-\epsilon$)-greedy policy, it uses the $\epsilon$ component as being based on the action that a
human would take, which we learn via a classifier from images (i.e., game frames) to actions. The
second application would be to run the standard DQN Q-learning algorithm, but run it on the
\emph{human data}, and see how it compares to the normal DQN. For both experiments, we use code
based on Nathan Sprague's open-source Python implementation of the Deep Q Learning
network\footnote{\url{https://github.com/spragunr/deep_q_rl}}.

As reported in~\cite{mnih-dqn-2015}, the DQN algorithm follows a $(1-\epsilon)$-greedy policy. A
policy is necessary because in active reinforcement learning, the agent needs to explore the state
space to learn an optimal policy. Their algorithm during training starts with $\epsilon=1$, so the
DQN agent picks random actions to start. Then $\epsilon$ is gradually reduced to 0.1 after the first
one million frames, and then fixed thereafter. The reason for high $\epsilon$ at the beginning is so
that the agent can sufficiently explore the state space, which is a requirement for Q-learning to
converge and find the optimal policy. Humans, however, do not generally perform random actions when
first starting to play a game, so it would be interesting to see if, instead of $\epsilon$
corresponding to random actions, it actually corresponds to an action humans chose. To pick a human
action, we can take the data from the human game log, which consists of frames and the corresponding
action. We can train a classifier to classify from images to actions, so that when the agent has a
frame, to pick an action, it picks the action recommended by the classifier. Since image
classification is best done with convolutional neural networks (CNNs), we use CNN code based on
CAFFE~\cite{caffe}.

The other idea, doing Q-Learning on the human data, is possible by following the same DQN algorithm,
except when sampling mini-batches for updating purposes, we just update it based off of a database
of human data, rather than the data that the network was exploring. Note that this means the agent
\emph{does not even have to explore}, because it already has data. Q-Learning is off-policy, so in
theory, it should still converge to some passable policy.

\textbf{TODO revise this ... and *hopefully* I can actually DO all of this stuff! I actually think
the Q-Learning on the robot might be very hard to finish in a short amount of time.}


%%%%%%%%%%%%%%%%%%%%%%%%%%%%%%%%%%%%%%%%%%%%%%%%%%%%%%%%%%%%%%%%%%%%%%%%%%%%%%%%
\section{Results}\label{sec:results}

\subsection{Human Experiment Results}\label{ssec:human_results}

\begin{figure}[t]
\begin{center}
\fbox{\rule[-.5cm]{0cm}{4cm} \rule[-.5cm]{4cm}{0cm}}
\end{center}
\caption{\textbf{TODO Put some boxplots or plots of human scores? Similar to what I had in the
presentation}. The yellow dot is the \emph{average} DQN score for that game, and the red dot is the
first author's \emph{average} score.}
\label{fig:human_results}
\end{figure}

We deployed our experiment on AMTurk in small sets, so as to not overload our server. We got a total
of \textbf{TODO} responses, which is less than our goal of 120, and in future work we plan on
finishing all 120.

First, we report on the raw scores players got. In Figure~\ref{fig:human_results}, we present three
boxplots of the players' \emph{highest} scores, for each of the three games. We also compare this
with the average DQN score (from the Nature paper), and the average score of the first author. (The
reason for using the highest of the players' scores is that they only have ten minutes, to provide a
reasonably fair comparison.) \textbf{TODO I might have to clarify this a bit.} We see that Pong was
especially difficult for the players. The score of Pong is calculated as the player score subtracted
from the computer score, so the scores range from $-21$ (i.e., losing 21 to 0) to $+21$. Breakout
was also difficult for them. A common complaint about Pong and Breakout was that the controls were
too difficult. Doing well in those games requires solid coordination in the controls, plus knowing
how to hit the balls diagonally. Breakout requires a fast reaction time, which is generally not a
problem for computer players. With Space Invaders, our players had a relatively easier time with the
controls, and consequently, were able to perform better, with our better players exceeding the
performance of DQN.

\textbf{TODO discuss transfer learning, briefly}

We also received some interesting comments about the game. Here is a sampling of them (all from
different players, and non-edited):

\begin{itemize}
    \item (Pong) \emph{my strategy during this game was an utter disaster throughout the entire
    phase of play.}
    \item (Pong) \emph{Breakout helped learn how to hit the ball at an angle}
    \item (Breakout) \emph{Try to chip out one column then get the ball to go up the column and
    bounce around the top, hitting all the high point blocks}
    \item (Space Invaders) \emph{Eliminate most off the lowest aliens first, then concentrate on an
    outer column}
    \item (All games) \emph{If there were a way to have better contorl of the paddle, I would try to
    hit more angular shots in the PONG game. The Atari paddle controller on the old 2600 system made
    this easily possible. The keyboard controls show weakness [...]}
\end{itemize}

It is indeed problematic that controls were an issue, but it is difficult to get around that.

\subsection{DQN Experiment Results}\label{ssec:dqn_results}

\begin{figure}[t]
\begin{center}
\fbox{\rule[-.5cm]{0cm}{4cm} \rule[-.5cm]{4cm}{0cm}}
\end{center}
\caption{\textbf{TODO Now compare human vs dqn results! This should be exciting}}
\label{fig:dqn_1}
\end{figure}

\begin{figure}[t]
\begin{center}
\fbox{\rule[-.5cm]{0cm}{4cm} \rule[-.5cm]{4cm}{0cm}}
\end{center}
\caption{\textbf{TODO Add a second plot with some further comparison. Ideally, I have two DQN plots,
both testing different things.}}
\label{fig:dqn_2}
\end{figure}

\textbf{TODO try to get ANYTHING up here!}

Due to lack of computational power at our disposal, we were unable to run the experiments for as
long as we wished.


%%%%%%%%%%%%%%%%%%%%%%%%%%%%%%%%%%%%%%%%%%%%%%%%%%%%%%%%%%%%%%%%%%%%%%%%%%%%%%%%
\section{Conclusions}\label{sec:conclusions}

\textbf{TODO elaborate on the importance of the results, once I get them!} 

Straightforward directions for future work include continuing the Amazon Mechanical Turk experiments
on more human participants, testing on other Atari games, and identifying which DQN settings to use
to maximize the benefit of human data. Some more elaborate extensions of this work might involve
running Q-Learning on the human data itself to see if the Q-Learner's policy is as good as what DQN
learned, and investigating inverse reinforcement learning to propose a pseudo-reward function that
explains people's behavior in their exploration stages.

%%%%%%%%%%%%%%%%%%%%%%%%%%%%%%%%%%%%%%%%%%%%%%%%%%%%%%%%%%%%%%%%%%%%%%%%%%%%%%%%
\section*{Acknowledgments}

We thank Andrew Liu for setting up the Amazon Mechanical Turk experiment.


%%%%%%%%%%%%%%%%%%%%%%%%%%%%%%%%%%%%%%%%%%%%%%%%%%%%%%%%%%%%%%%%%%%%%%%%%%%%%%%%

\addtolength{\textheight}{-12cm}
% This command serves to balance the column lengths on the last page of the document manually. It
% shortens the textheight of the last page by a suitable amount.  This command does not take effect
% until the next page so it should come on the page before the last. Make sure that you do not
% shorten the textheight too much.

\bibliographystyle{IEEEtran}
\bibliography{Daniel_Seita_Final_Paper_CS287}

% Daniel: This was their old table code.
%\begin{table}[h]
%\caption{An Example of a Table}
%\label{table_example}
%\begin{center}
%\begin{tabular}{|c||c|}
%\hline
%One & Two\\
%\hline
%Three & Four\\
%\hline
%\end{tabular}
%\end{center}
%\end{table}

% Daniel: This was their old figure code.
%\begin{figure}[thpb]
%\centering
%\framebox{\parbox{3in}{We suggest that you use a text box to insert a graphic (which is ideally a
%300 dpi TIFF or EPS file, with all fonts embedded) because, in an document, this method is somewhat
%more stable than directly inserting a picture.  }}
%%\includegraphics[scale=1.0]{figurefile}
%\caption{Inductance of oscillation winding on amorphous
%magnetic core versus DC bias magnetic field}
%\label{figurelabel}
%\end{figure}


\end{document}
